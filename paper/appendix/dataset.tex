\section{Appendix -- Dataset Details}
The following details in this appendix are for clarification purposes.
All examples in this appendix are created via PySpark using dataframes.

\subsection{Using PySpark to Load Data}
All example code was created using Jupyter Notebooks creating a PySpark session. The details of setting up such session are below in the code example. All subsequent details are derived from the \texttt{sc}\footnote{sc = Spark Session}.
\begin{minted}[fontsize=\small]{python}
# General Imports
import glob
import shutil

# Setup Spark Session
from pyspark.sql import SparkSession

# Get Spark Functions Needed
from pyspark.sql.functions import col, udf, split, explode

# Get Datatypes needed for DataFrame manipulation
from pyspark.sql.types import IntegerType, StringType

# Setup Spark Session
sc = SparkSession \
        .builder \
        .master("local[*]") \
        .appName("data_clean_up") \
        .getOrCreate()

# Print Spark Version being run
print("Spark V: ", sc.version)
\end{minted}
\clearpage


\subsection{Details on English Dataset}
\label{appendix:dataset_english}
The English dataset is based on tweets\footnote{Tweets from Twitter} that are saved as a \texttt{csv}\footnote{csv = Comma Separated Values} file. The following code was use to filter and process the file for use in this project.
\begin{minted}[fontsize=\small]{python}
# *********************************
# *** English Data Preparations ***
# *********************************

# Load English CSV File into a Dataframe
df_english = sc.read.csv("data/english.csv", header=True, inferSchema=True)
print(f'Count (raw): {df_english.count()}')

# Print Columns
print(f'English Source Columns: {df_english.columns}\n')

# Filter only lavbels that we mathc with the German counter parts
filter_values = ["love", "hate", "neutral", "anger", "happiness", "surprise", "sadness", "worry", "enthusiasm"]
df_english_filtered = df_english.filter(col("sentiment").isin(filter_values))

# Remove unnecessary column
df_english_filtered = df_english_filtered.drop("tweet_id")

# Rename Columns
df_english_filtered = df_english_filtered.withColumnRenamed("sentiment", "emotion")
df_english_filtered = df_english_filtered.withColumnRenamed("content", "sentence")

# Make sure the column order to the same for both german and english csv files
df_english_filtered = df_english_filtered.select('sentence', 'emotion')

print(f'Count (filtered): {df_english_filtered.count()}')

# Group By for Details & Count
df_english_grouped = df_english_filtered.groupBy('emotion').count()

# Show Groupings and Respetive Counts
print('\nGrouped & Count by "emotion"')
df_english_grouped.show(truncate=0)

# Save Dataframe to CSV
directory_path = 'data/spark_data_parts'
df_english_filtered.coalesce(1).write.csv(directory_path, header=True, mode="overwrite")

file_pattern = 'part-00000*.csv'
file_path = glob.glob(directory_path + '/' + file_pattern)[0]

shutil.move(file_path, './data/data_en.csv')
\end{minted}
\clearpage

\subsection{Details on German Dataset}
\label{appendix:dataset_german}
The German dataset was downloaded from the ETH's\footnote{ETH (German: Eidgenössische Technische Hochschule) or Swiss Federal Institute of Technology} servers via "link" as a \texttt{JSON} file. Since the the file was \texttt{JSON} we will need to filter and explode the file based on teh emotions that are attached to it.
\begin{minted}[fontsize=\small]{python}
# ********************************
# *** German Data Preparations ***
# ********************************

# Load German JSON File into a Dataframe
df_german = sc.read.json("data/german.json")
print(f'Count (raw): {df_german.count()}')

# Print Columns
print(f'German Source Columns: {df_german.columns}\n')


# We Need to "split" the "artice_emotion" column since the ETH team listed
# multiple emotions in one column
# In order to "explode" the column to it's distinct rows
df_german_exploded = df_german.select('*', explode('article_emotion').alias('emotion'))

# Rename Column
df_german_exploded = df_german_exploded.withColumnRenamed("title", "sentence")

# Remove unnecessary column
df_german_exploded = df_german_exploded.drop("article_id")
df_german_exploded = df_german_exploded.drop("article_stance")
df_german_exploded = df_german_exploded.drop("paragraphs")
df_german_exploded = df_german_exploded.drop("source")
df_german_exploded = df_german_exploded.drop("article_emotion")
df_german_exploded = df_german_exploded.drop("snippet")

# Make sure the column order to the same for both german and english csv files
df_german_exploded = df_german_exploded.select('sentence', 'emotion')

print(f'Count (filtered): {df_german_exploded.count()}')

# Group By for Details & Count
df_german_grouped = df_german_exploded.groupBy('emotion').count()

# Show Groupings and Respetive Counts
print('\nGrouped & Count by "emotion"')
df_german_grouped.show(truncate=0)


# Save Dataframe to CSV
directory_path = 'data/spark_data_parts'
df_german_exploded.coalesce(1).write.csv(directory_path, header=True, mode="overwrite")

file_pattern = 'part-00000*.csv'
file_path = glob.glob(directory_path + '/' + file_pattern)[0]

shutil.move(file_path, './data/data_de.csv')
\end{minted}


\clearpage
